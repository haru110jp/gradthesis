\documentclass[a4paper]{jarticle}
\usepackage{amsmath,amsthm,amssymb}
\usepackage[dvipdfmx]{graphicx}
\usepackage{here}
\usepackage{ascmac}
\usepackage{url}
\usepackage{natbib}
\newtheorem{teiri}{定理}
\newtheorem{hodai}{補題}

\title{都市環境整備の条件及びその都市集積への影響に関する空間経済学的分析\thanks{\textbf{重要:}現段階での本稿における分析は筆者が独力で行ったもので、未熟なものにとどまっている。そのため、論文の体裁にまとめてあるものの今後の研究の方向性を定める研究計画書としての価値がほとんどであることをお断りしておく。研究計画書として読む場合は、上の「概要」と第8節の「結論と今後の展望」だけに目を通していただたければ十分である。本稿の間違いはすべて山岸に帰属する。忌憚のないコメント、アドバイスをいただけると幸いである。}}
\author{山岸 敦\thanks{東京大学経済学部経済学科4年 尾山ゼミ及び松井ゼミ所属}}

\date{2015/4/10}

\begin{document}

\maketitle

\begin{abstract}
大都市中心部の環境がその周縁部に比べてよく整備され、大規模な緑地、公園が立地している現象は日本のみならず世界各国で観察される。しかし、都市中心部は土地価格が高いため、住宅、商業用地として最大限活用し地代収入を得る機会を放棄してまで都市中枢部の設計に余裕を持たせたり、公園を作ったりする理由やその効果は自明ではなく経済学的な説明を要する。本研究は\citet{ottaviano02:aggl}を拡張し都市中心の環境が改善され、維持される条件及びその空間経済学的帰結を分析する。まず、Central Business District(CBD)への通勤を導入した\citet{tabuchi98:urba}流のモデルに良好な環境への定期的なアクセスがもたらす効用を加えることで都市中心の環境整備が行われる条件を示す。次に、それを\citet{ottaviano02:aggl}の2都市モデルに埋め込み、その空間経済学的インプリケーションを分析する。その結果、都市中心の環境整備は集積力として働くこと、各都市の自発的な都市環境整備が時に全体として非効率的な集積をもたらしうること、総余剰が改善するかどうかはパラメーターに大きく依存することが示される。モデルの挙動を完全に特徴づけること、近視眼的な行動改訂ルールをforward-lookingなものに改めること、都市環境整備の効果についての実証研究を行い仮定の妥当性を検討すること、より事態の説明に適したモデルを探索することなどが考えうる今後の研究方針として挙げられる。

\end{abstract}

\newpage{}

\section{introduction}
都市中心部で大規模かつ包括的な環境整備が行われる例は世界中に無数にある。例えば、東京には渋谷や代々木、新宿といった土地価格の高い地域に多数の大規模な緑地が存在し、ニューヨークにはマンハッタンにセントラルパークが150年以上にわたって存在する。また、都市中心部はその周縁部に比べて逆に空間的に余裕のある設計になっていて、街路樹などがよく整備されている例も多い。しかし、大都市中心部の土地は一般に非常に高いことを考慮すれば、限界まで住居などに土地を利用した方が金銭的収入を引き上げることができるため、家賃収入を放棄してまでこうした空間が都市中心部に建設され長期にわたって維持されてきた現象は自明とは言えず説明を要する。さらに、こうした都市中心部の緑地が空間的立地構造及び経済厚生に与える影響を分析することは都市政策において政策的含意が大きいと考えられる。

本論文ではこうした課題を分析するために、良好な自然、都市環境への安定的なアクセスがもたらす効用、或いはそれへの阻害のもたらす不効用を仮定し、それを\citet{tabuchi98:urba}や\citet{ottaviano02:aggl}流のモデルに加えて拡張することで、単一都市モデルにおいてCentral Business District(CBD)、すなわち都市中枢部の環境が整備される条件を示す。続いて、これを\citet{ottaviano02:aggl}の2都市の新経済地理学モデルに埋め込み、都市中心部環境整備の空間経済学的インプリケーションを分析する。新経済地理学的なモデルを用いて環境問題、環境政策を分析した先行研究として、コンパクトシティーの環境負荷について人口移動まで考慮して分析した\citet{gaigne12:arec}があるが、都市環境整備政策について新経済地理学モデルを用いて人口移動をも考慮して分析した研究は筆者の知る限り存在しない。しかし、途上国に多数の都市が生まれ都市環境問題の深刻化が現在も進行していること、先進国の多くの都市中心部で環境に配慮した都市づくりがなされていること、などの実情を踏まえるとこの問題に経済学的にアプローチする価値は大きいと考える。

なお、上記の環境コストの存在やその定式化に対する仮定の妥当性は当然実証的に検討されるべきであるし、仮に存在するにしても量的な効果の大きさもまた実証的に計測されねばならない。この課題に対する包括的な研究は筆者の知りうる限り現在までのところ存在しないが、こうした効果の存在を示唆する研究として自然環境への接触が幸福感を増進させることを実験的に示した\citet{berman08:cogn}や緑地の多い地域への移住が幸福感を増大させることをパネルデータ分析で示した\citet{alcock14:long}などがあり、主に心理学の領域で活発に議論が交わされている。今後は経済学的な観点からの分析の充実も強く望まれる。筆者も、現在このような実証研究の構想を練っているがそれについては第8節に譲る。

本論文は、以下のように構成される。第2節でベースモデルとなる単一都市モデルを用いて、都市中心部の環境整備の条件を導く。第3節から第5節で\citet{ottaviano02:aggl}の枠組みを導入する。第6節でそれに第2節のモデルを埋め込み、本論文の主要な結果を得る。第7節はモデルの挙動を例示する数値計算を紹介する。第8節で結論と今後の研究方針について議論する。

\section{単一都市モデルにおける都市中枢部環境整備}
ベースモデルとして、\citet{tabuchi98:urba}のようにCBDへの通勤費用を考慮することで地代勾配と都市生活費用を考えるモデルに自然環境へのアクセス阻害による不効用を導入する。ただし、\citet{ottaviano02:aggl}や\citet{combes08:econ}、\citet{sato11:spat}の簡略化されたバージョンを基本として説明する。なお、以下の分析では準線形の効用関数を仮定し、効用水準は基準財により表示されているとする。

直線の土地上にCBDがあり、一般性を失うことなくここを座標0と置く。労働者の数はLであり、皆がCBDへ通勤する。労働者はみな1単位の土地を消費するとすれば、都市空間は座標$[-\frac{L}{2}, \frac{L}{2}]$に均質に広がっている。都市の外側には農村が広がっており、農地地代を一般性を失うことなく0に基準化する。労働者の住居の座標を$x$と表したとき、線形の通勤費用を想定して、通勤費用は$\theta_c |x|$とする。

$\theta_c$は単位距離あたりの通勤費用である。また、地点$x$の地代は$R(x)$で表す。地代収入は、都市の住民に均等分配されると仮定する。\footnote{この仮定は「公的所有」の仮定と呼ばれる。代表的な代替的な仮定としては、「不在地主」の仮定がある。}

次に、自然環境へのアクセス阻害による不効用と、都市中枢環境整備の効果を定式化する。簡略化のためCBDの環境を整備する際の追加的な土地の消費量はゼロとする。地点$x$の労働者が被るコストを$D(x)$と表すとき
\begin{equation}
  \begin{split}
  D(x) = & \theta_g (\frac{L}{2} - |x|) \ \ if\ there\ is\ not\ a\ park\ in\ CBD \\
         = &- \theta_g \frac{L}{4} + \theta_g |x - \frac{L}{4}| \ \ if\ there\ is\ a\ park\ in\ CBD
  \end{split}
\end{equation}
と定式化する。ただし、$\theta_c > \theta_g$を仮定する。(1)式は次のようなことを意味する。もしCBDの環境が改善されていないなら、直近の農地、すなわち都市の両端に自然が残っていると考え、自然へのアクセスのための時間、金銭的負担やCBDに近づくにつれての都市環境悪化に伴い居住地からそこまでの移動距離に比例したコストを負うと考える。その一方で、CBDの環境を整備した場合にはその付近に住む人間は良好な環境へのアクセスを獲得し、結果としてCBDと都市の臨界点との中点でもっとも環境コストが高くなる\footnote{この定式化が、事態をうまく表している保証はない。よりよい仮定の探索は継続するつもりである。}。2つコメントを加える。第一に、本来であれば中心地以外の環境を整備する方が効用を改善できる可能性を検討しなければならないが、このモデルではCBDの特殊性によりこの可能性を事実上捨象している\footnote{例えば設置する緑地の数を1つに絞らない、などの変更を加えた時中央以外の環境整備の可能性は重要性を増すかもしれない。この部分は、今回はいったん深く立ち入らないで単純にCBDの環境を改善するか否かの二項問題に単純化した。}。第二に、この定式化によれば、受益者の内で良好な環境、緑地への距離が近いほど環境整備に対するWillingness To Payが上昇する\footnote{これは\citet{salazar07:esti}による大きのな公園建設に関する実証研究と整合的である。}。

まず、環境整備前の均衡での$x$地点での家賃$R(x)$を求める。均衡では労働者の移住行動は起こらないことから、$x \in [-\frac{L}{2}, \frac{L}{2}]$において、どの地点$x$に住む労働者も効用水準は一定となる。すなわち、
\[
\theta_c |x| + \theta_g (\frac{L}{2} - |x|) + R(x) = constant \ \ \forall x \in [-\frac{L}{2}, \frac{L}{2}]
\]
が成立し、これより
\begin{equation}
R(x) = (\theta_c - \theta_g)(\frac{L}{2} - |x|)
\end{equation}
が成立する。これより、平均地代額は$\frac{L(\theta_c - \theta_c)}{4}$であり、これが公的所有の仮定より各労働者に還元される金額であるから、都市生活の総費用は
\begin{equation}
\frac{L(\theta_c + \theta_g)}{4}
\end{equation}
と求められる。

次に、環境整備後の均衡での地代及び都市生活の費用を求める。先ほど同様、都市生活のコストはどの地点でも同一になるのが均衡であることを利用すると、$x$が正の場合(1)より多少ノーテーションを乱用すれば
\begin{align}
&R(x) =  -(\theta_c + \theta_g)|x| + \frac{L}{2}\ \  ,0 \leq x \leq \frac{L}{4} \\
&R(x)      =  -(\theta_c - \theta_g)|x| + \frac{L}{2} (\theta_c - \theta_g) \ \ , \frac{L}{4} \leq x \leq \frac{L}{2}
\end{align}
が得られる。$x$が負の場合も対称性より同様の式が導かれる。注目すべきは、最初はCBDから離れるに連れて急激に地代が下がるが、ある点から傾きがゆるやかになることである。この地代分布パターンは後半に観察されることが知られており、東京圏でも確認されていることである~\cite{kanemoto97:urba}\footnote{ただし、この事実に対する都市経済学での標準的説明とは異なる。それでも、既存の理論の仮定を抜きにしてもこの構造が現れることが示されたことは興味深い。}。

ここで、環境整備、維持のための費用を定式化する。ここでは、単純化のため環境整備後の地点0での地代の関数として、$C(R(0))$が総費用としてかかると定式化する。ただし、$C(・)$は連続な狭義増加関数であるとする。さらに、人口が大きければそれに比例した規模の環境整備が必要である、すなわち、数学的には$C(R(0)) = C(\frac{\theta_c L}{2})$は一次同次と仮定する。これにより一人あたりの費用は$\frac{C(\theta_c)}{2}$となる。地代収入は均等に配分されることを思い出せば、環境整備を行った時の都市生活の総費用は
\begin{equation}
\frac{L \theta_c}{4} + \frac{L \theta_g}{8} + \frac{C(\theta_c)}{2}
\end{equation}
となる。(3)と(6)を比較することにより、CBDの緑地化が行われる必要十分条件は
\begin{equation}
\frac{L\theta_g}{8} >\frac{C(\theta_c)}{2}
\end{equation}
となる。

\begin{teiri}
CBDの環境整備が行われるための必要十分条件は、$\frac{L\theta_g}{8} >\frac{C(\theta_c)}{2}$である。
\end{teiri}

定理1より、「都市規模が大きい」、「環境コストが大きい」、「所与の家賃に対する環境整備費用が小さい」、「通勤費用が小さく、都市部の家賃が安い」ときに都市中枢の環境整備がなされやすいことがわかる。

\section{新経済地理学モデルへの拡張}
本節では、前節のモデルを\citet{ottaviano02:aggl}の枠組みに埋め込むことで、都市環境整備の集積への影響の分析と厚生評価を行う。都市での費用構造以外は\citet{ottaviano02:aggl}.と同一であるため、ここでは簡単に導入する。本説の説明は\citet{ottaviano02:aggl}および\citet{combes08:econ}に準拠している。より詳細な説明に関してもこれらの文献を参照していただきたい。

\subsection{モデル}
経済は直線上に立地する都市1、都市2の2つからなる。両者は十分離れていて重なる部分はないとする。生産要素として、移動不可能な「農家」Aと移動可能な「労働者」Bを想定する。$\lambda$はLのうち地域1に存在する割合を示し、Aは両地域に均等に分布しているとする。

この経済には2つの財がある。ひとつは、「農業財」であり、Aを1単位用いると1単位生産され、収穫一定である。また、農業財の取引に輸送費はかからないため、これを価値基準財とする。もう一つは水平的に差別化された材(「工業材」)であり、Lのみを用いて収穫逓増、不完全競争の下で生産される。工業財を生産する工場は連続的にN存在すると仮定する。\footnote{このとき、企業数と剤の数は一対一対応する。}それぞれの財は、一単位輸送するときに$\tau$だけ輸送費がかかる\footnote{これは\citet{krugman91:incr}以来一般的な想定である氷解型輸送とは異なり、貿易される財の価値に依存しない。}

人々の選好はみな同一で、次の2次形式のsubutilityを持つ準線形効用関数で表せるとする。
\begin{multline}
U(q_0; q(i), i \in[0, N]) = \\
\alpha \int_{0}^{N}q(i) di - \frac{\beta - \gamma}{2}\int_{0}^{N} (q(i))^2 di - \frac{\gamma}{2}(\int_{0}^{N}q(i)di))^2 + q_0
\end{multline}
ただし、$q(i)$は財iの消費量を、$q_0$は基準財(農業財)の消費量を表す。パラメーター$\alpha>0$は工業材への選好の度合いを、$\beta > \gamma > 0$については、$\beta$は多様性への選好を、$\gamma$は代替性を示す。\footnote{この事については、\citet{ottaviano02:aggl}のAppendixでわかりやすい例示がなされている}

各個人は初期保有$\overline{q}_0$とAないしLを一単位を保有する。それへの給与を$y$と表記すれば、予算制約式は
\begin{equation}
\int_{0}^{N} p(i)q(i)di + q_0 = y + \overline{q}_0
\end{equation}
と書ける。ただし、p(i)は財iの価格を表し、農業財価格は1に基準化した。$\overline{q}_0$は十分に大きく、均衡において常に農業財の消費量が正になるように仮定する\footnote{この仮定は農業財、工業材両方をつねに好む、という事実と整合的である。}。これにより、内点解にのみ焦点を絞ることができる。

(8)の下で(7)を最大化すると、財iへの需要は
\begin{equation}
q(i) = a - bp(i) + c \int_{0}^{N}(p(j) - p(i))dj
\end{equation}
ただし、$a \equiv \frac{\alpha}{\beta + (N - 1)\gamma}$、$b \equiv \frac{1}{\beta + (N - 1)\gamma}$、$c \equiv \frac{\gamma}{(\beta - \gamma) (\beta + (N - 1)\gamma)}$である。これを用いると、間接効用関数は
\begin{multline}
V(y;p(i), i \in[0, N]) = \frac{a^2N}{2b} - a \int_{0}^{N}p(i)di + \\
 \frac{b + cN}{2}\int_{0}^{N}(p(i))^2di - \frac{c}{2}(\int_{0}^{N}p(i)di)^2 + y + \overline{q}_0
\end{multline}
と表せる。

供給サイドに視点を移す。農業財生産の収穫一定及び輸送費ゼロの仮定から、農家の賃金は1になる(農業財が基準財になっていることに注意)。また、工業製品の生産技術は、「$\phi$単位のLを消費すれば、限界費用0で生産できる」というものを仮定する。これは、収穫逓増の1つの表現になっている。$\phi$は収穫逓増の度合いを示す。

労働市場の均衡条件から、地域1、2の企業数を$n_1、n_2$と置くとき、
\begin{align}
&n_1 = \frac{\lambda L}{\phi} \\
&n_2 = \frac{(1 - \lambda) L}{\phi} \\
&N = \frac{L}{\phi}
\end{align}
が成立する。ここで、企業の対称性から地域1のある代表的企業の地域1での人口一人あたり需要を$q_{11}$、地域1の企業の地域2での人口一人あたり需要を$q_{12}$と定義し、そこでの販売価格をそれぞれ$p_{11}、p_{12}$とおくことができる。同様にして$q_{22}、q_{21}、p_{22}、p_{21}$を定義できるが、以下では対称性を利用し一般性を失うことなく地域1に分析を絞る。(9)より、
\begin{align}
&q_{11} = a - (b + cN)p_{11} + cP_1 \\
&q_{12} = a - (b + cN)p_{12} + cP_2 \\
\intertext{ただし、$P_1、P_2$はそれぞれその地域での価格指数で}
&P_1 = n_1 p_{11} + n_2 p_{21} \notag \\
&P_2 = n_1 p_{22} + n_2 p_{12} \notag 
\end{align}
である。最後に、地域1のある企業の利潤$\pi_1$は
\begin{equation}
\pi_1 = p_{11}q_{11}(\frac{A}{2} + \lambda L) + p_{12}q_{12}(\frac{A}{2} + (1 - \lambda) L) + \phi w_1
\end{equation}
である。$\frac{A}{2}$は地域1に住む農家の数で、$w_1$は地域1の労働者の賃金である。

\subsection{短期価格均衡}
ここでは、移住行動を考えず$\lambda$を所与とした時の均衡である短期価格均衡を導出する。企業は価格を操作し(16)を最大化する。(14)、(15)は均衡価格においても成立することを利用すれば、均衡価格は
\begin{align}
p_{11} = &\frac{1}{2}\frac{2a + \tau c (1 - \lambda) N}{2b + cN} \\
p_{22} = &\frac{1}{2}\frac{2a + \tau c (\lambda) N}{2b + cN} \\
p_{12} = &p_{22} + \frac{\tau}{2} \\
p_{21} = &p_{11} + \frac{\tau}{2}
\end{align}
である。これらがどの$\lambda$に対しても正になるには
\begin{equation}
\tau < \tau_{trade} \equiv \frac{2 a \phi}{2 b \phi + cL}
\end{equation}
が必要十分である。以下、常にこれを仮定する。まず、労働者の所得は、均衡において企業の利潤がゼロになることから
\begin{multline}
w_1(\lambda) = \frac{b \phi + c L}{4(2 b \phi + cL)^2 \phi^2}((2a\phi + \tau c L (1 - \lambda))^2(\frac{A}{2} + \lambda L) \\
+ (2 a \phi - 2 \tau b \phi - \tau c L (1 - \lambda))^2(\frac{A}{2} + (1 - \lambda)L)) 
\end{multline}
となる。これと均衡価格を用いて間接効用関数(10)を評価すると、地域1の労働者の間接効用$V_1(\lambda)$が得られる。

\section{移住行動}
この節では、労働者が移住し$\lambda$が変化する状況を考える。ここで、移住行動は近視眼的であり、現時点での効用の差にのみ依存するとする。すなわち、$\Delta V(\lambda) = V_1(\lambda) - V_2(\lambda)$が正ならば地域1への移住が進み、逆は逆となる。\citet{ottaviano02:aggl}より、
\begin{align}
&\Delta V(\lambda) = C \tau(\tau^* - \tau)(\lambda - \frac{1}{2})
\intertext{ここで}
&C \equiv (2 b \phi (3 b \phi + 3 c L + c A) + c^2L(A + L))\frac{L(b \phi + c L)}{2 \phi^2(2 b \phi + c L)^2} \notag \\ 
& \tau^* \equiv \frac{4 a \phi (3 b \phi + 2 c L)}{2 b \phi(3 b \phi + 3 c L + c A)+ c^2 L (A + L)} \notag
\end{align}
となる。(23)式は、都市のコストを考慮する以前の効用格差であり、この特徴付けは\citet{ottaviano02:aggl}のProposition 1に詳しい。後の節でこれに都市の生活コストを加える事で、都市環境整備の集積への影響を分析する。

\section{社会的な最適集積水準}
社会全体の総効用$W(\lambda)$は、次の式で与えられる。
\begin{equation}
W(\lambda) = \frac{A}{2}(S_1(\lambda) + 1) + \lambda L (S_1(\lambda) + w_1(\lambda)) + \frac{A}{2}(S_2(\lambda) + 1) + (1-\lambda)L(S_2(\lambda) + w_2(\lambda)) \notag
\end{equation}
ここで、$S_1、S_2$は地域1、2の消費者余剰である。\footnote{具体的な形は\citet{ottaviano02:aggl}を参照すること}社会的に最適状況では限界費用が価格に等しくなっている、つまり$p_{11} = p_{22}$ かつ$p_{12} = p_{21} = \tau$が成立している。これと企業のゼロ利潤条件より、$w_1 = w_2 = 0$を得る。以上を代入して整理すれば
\begin{align}
&W(\lambda) = C^o \tau (\tau^o - \tau) \lambda (\lambda - 1)L + constant
\intertext{ただし、}
&C^o \equiv \frac{L}{2 phi^2}(2 b \phi + c (A + L)) \notag \\
&\tau^o \equiv \frac{4 a \phi}{2 b \phi + c (A + L)} \notag
\end{align}
となる。後の節ではこれに都市の生活コストを加える事で分析を拡張する。都市コストを考慮しない(24)式の特徴付けについては、\citet{ottaviano02:aggl}のProposition 2を参照してほしい。

\section{都市コストの導入}
本説では、2節のモデルで分析した都市の生活コストを3-5節で考えた\citet{ottaviano02:aggl}のモデルに埋め込む。すなわち、3節のモデルの都市内部で2節のモデルが動き、都市での生活費用が確定される。地域1の人口は$\lambda L $であり、地域2のそれは$1 - \lambda$であることからそれを(3)式、(5)式に代入することで環境整備をしなかった時の一人あたりのコスト及びした時のコストが求まる。なお、特に言及がない限り、「両方の都市が環境整備をした/していない場合」について考察する。\footnote{片方の都市しか環境整備をしない状況の分析は道半ばであり、今回は省略する。しかし、一定の成果が得られ次第含めるつもりである。現段階で言える重要なことは、片方のみが都市環境整備をするとき(7)式より初期分布が非対称になっていて、より集積している方が環境整備を行うということである。}

\subsection{都市コストの集積への影響}
まず、環境整備をしなかった場合について考える。都市コストの差は
\[
\frac{(\theta_c + \theta_g) \lambda L }{4} - \frac{(\theta_c + \theta_g) (1 - \lambda) L }{4} =  L(\lambda - \frac{1}{2})\frac{\theta_c + \theta_g}{2}
\]
より、これを(23)式から引くことで、都市コストを考慮した効用の差が
\begin{equation}
-C(\tau^2 - \tau^* \tau + \frac{(\theta_c + \theta_g)L}{2C})(\lambda - \frac{1}{2})
\end{equation}
と求まる。この時、近視眼的行動改訂を仮定すると$\tau^2 - \tau^* \tau + \frac{(\theta_c + \theta_g)L}{2C}$が負なら集積均衡($\lambda = 0 \ or \ 1 $)のみが、正なら分散均衡($\lambda = \frac{1}{2}$)のみが均衡となることが容易に示せる。よって、2次方程式の解の配置の問題に帰着する。判別式が正の時、上の式の解より
\[
\underline{\tau} = \frac{\tau^* - \sqrt{(\tau^*)^2 - \frac{2 (\theta_c + \theta_g)L}{C}}}{2} \\
\overline{\tau} = \frac{\tau^* + \sqrt{(\tau^*)^2 - \frac{2 (\theta_c + \theta_g)L}{C}}}{2}
\]
よって次の定理を得る。

\begin{teiri}
都市環境整備はなされていない状況を考える。もし$2(\theta_c + \theta_g)L < C(\tau^*)^2$ならば、$\underline{\tau} < \tau < \overline{\tau}$のとき、またその時に限り集積が唯一の安定均衡になり、$\tau = \underline{\tau}\ or \ \overline{\tau}$の時はどんな分布も不安定均衡になる。もし$2(\theta_c + \theta_g)L > C(\tau^*)^2$ならば、分散が唯一の均衡になる。
\end{teiri}

つづいて、都市環境整備がなされている場合についても同様な分析を行う。(5)式より、地域1と地域2でのコストの差は
\[
(\lambda - \frac{1}{2})(\frac{\theta_c L}{2} + \frac{\theta_g L}{4})
\]
となり、これを(23)式から引くことで効用の差は
\begin{equation}
-C(\tau^2 - \tau^* \tau + \frac{\theta_cL}{2C} + \frac{\theta_g L}{4C})(\lambda - \frac{1}{2})
\end{equation}
と求まる。先ほどと同様に2次方程式を解くことで、
\[
\underline{\tau}_e = \frac{\tau^* - \sqrt{(\tau^*)^2 - \frac{2 \theta_cL}{C} - \frac{\theta_g L}{4} }}{2} \\
\overline{\tau}_e = \frac{\tau^* + \sqrt{(\tau^*)^2 - \frac{2 \theta_cL}{C} - \frac{\theta_g L}{4} }}{2}
\]
これより、次の定理を得る。

\begin{teiri}
都市環境整備はなされていない状況を考える。もし$2\theta_cL < C(\tau^*)^2$ならば、$\underline{\tau}_e < \tau < \overline{\tau}_e$のとき、またその時に限り集積が唯一の安定均衡になり、$\tau = \underline{\tau}_e\ or \ \overline{\tau}_e$の時はどんな分布も不安定均衡になる。もし$2\theta_cL > C(\tau^*)^2$ならば、分散が唯一の均衡になる。
\end{teiri}

定理2と3を比較すると、$\theta_g$の存在により都市環境を整備した時としない時とで集積をもたらす輸送費の範囲が変化していることがわかる。都市環境を整備していない時のほうがその範囲が狭く、都市環境を整備することでそれを拡大することができるから、都市環境整備は集積力として働くことがわかった。

また、この結果により、自発的に各都市が環境整備を行うことで、$\tau$の値によっては都市の集積分散構造が変化する可能性が示された。例えば、現在都市は分散($\lambda = \frac{1}{2}$)しているとしよう。外生的な要因によって環境整備費用$C(\theta_c)$が十分低下すると、(6)式より各年は自発的に都市環境を整備する。しかし、これは定理2、3より集積をもたらす$\tau$の範囲を変えるため、結果として都市環境が空間的立地に影響しうるのである。

\subsection{都市コストと社会的最適集積水準}
続いて、都市コストを導入した場合の社会的最適水準について考える。ここでは、社会的総余剰の最大値を求めるために\citet{ottaviano02:aggl}に従い、社会計画者は①労働者を地域間で自由に配分することができ、②価格を限界費用に一致させることによる企業の損失を一括所得移転を行うことで補うことができる、と想定する。まず、都市環境整備をしなかった場合(3)式より地域1の都市コストの合計は$\frac{1}{4}(\lambda L)^2 (\theta_c + \theta_g)$、地域2のそれは$\frac{1}{4}((1 - \lambda) L)^2 (\theta_c + \theta_g)$であることから、これらを(24)に合算し整理すると、
\begin{equation}
W(\lambda) = C^o L \lambda (\lambda - 1) (\tau^2 - \tau^o \tau + \frac{L}{2C^o}(\theta_c + \theta_g)) + const.
\end{equation}
である。右辺第一項は$\lambda$の2次方程式であり、判別式が正になることを仮定してこの式を解けば
\begin{equation}
\underline{\tau^0} = \frac{\tau^o - \sqrt{(\tau^o)^2 - \frac{2L(\theta_c + \theta_g)}{C^o}}}{2} , \overline{\tau^0} = \frac{\tau^o + \sqrt{(\tau^o)^2 - \frac{2L(\theta_c + \theta_g)}{C^o}}}{2}
\end{equation}
この係数の正負に注目することで次の定理を得る。

\begin{teiri}
都市環境整備がなされていない状況を考える。$(\tau^o)^2 > \frac{L}{2C^o}(\theta_c + \theta_g)$を仮定する。このとき、$\underline{\tau^o} < \tau < \overline{\tau^o}$なら、集積が最適である。$\tau = \underline{\tau^o}\ or \ \overline{\tau^o}$なら、どの分布も最適である。それ以外の場合なら分散が最適である。
\end{teiri}

同様にして都市環境整備をした場合を考える。(5)式より地域1の都市コストの合計は$\frac{1}{4}(\lambda L)^2 \theta_c + \frac{1}{8}(\lambda L)^2 + \frac{1}{2}C(\theta_c)$、地域2のそれは$\frac{1}{4}((1 - \lambda) L)^2 \theta_c +  + \frac{1}{8}((1 - \lambda )L)^2 + \frac{1}{2}C(\theta_c)$であることから、これらを(24)に合算し整理すると、
\begin{equation}
W(\lambda) = C^o L \lambda (\lambda - 1) (\tau^2 - \tau^o \tau + \frac{L}{2C^o}\theta_c + \frac{L}{4C^o}\theta_g) + const.
\end{equation}
である。右辺第一項は$\lambda$の2次方程式であり、判別式が正になることを仮定してこの式を解けば
\begin{equation}
\underline{\tau^0}_e = \frac{\tau^o - \sqrt{(\tau^o)^2 - \frac{2L\theta_c}{C^o} - \frac{L \theta_g}{C^o} }}{2} , \overline{\tau^0}_e = \frac{\tau^o + \sqrt{(\tau^o)^2 - \frac{2L\theta_c}{ C^o} - \frac{L \theta_g}{C^o} }}{2}
\end{equation}
この係数の正負に注目することで次の定理を得る。

\begin{teiri}
都市環境整備がなされている状況を考える。$(\tau^o)^2 > \frac{2L}{C^o}\theta_c + \frac{L}{C^o}\theta_g$を仮定する。このとき、$\underline{\tau^o}_e < \tau < \overline{\tau^o}_e$なら、集積が最適である。$\tau = \underline{\tau^o}_e\ or \ \overline{\tau^o}_e$なら、どの分布も最適である。それ以外の場合なら分散が最適である。
\end{teiri}

定理4と5を比較することにより、環境整備によって都市集積が社会的に最適な範囲が拡大していることがわかる。これは、直感的には次のような理由による。大都市では都市の範囲が大きくなり、その分環境コストが増大する。環境整備はこのコストを社会全体から一定の支出を犠牲にして一部取り払うので、この大都市集積が不利になる要因を取り除くことができるからである。

\section{数値例}
ここで、実際にモデルのパラメーターに数字を代入し、モデルの挙動を例示する予定(プログラムのアップデートが間に合わず…)。

\section{結論と今後の展望}
都市中心部の環境整備の誘引や、その集積構造、社会厚生への影響を調べるための研究を企画した。まず、単一都市の状況をベースモデルとして都市が自発的にその中心部の環境を整備する条件を導きその特徴付けを行った。さらに、それを\citet{ottaviano02:aggl}の新経済地理学モデルに埋め込み、解析的な結果として都市環境整備は集積力を持つ、自発的な都市環境整備は時に非効率的な集積を促し社会厚生を損なう可能性がある、といった知見が得られた。

今後の研究方針として、第一により綿密にモデルを特徴付けることがあげられる。特に、初期状態が不完全集積の場合、(6)式は人口の差があることから集積している方の都市だけが環境整備を行う可能性があることを示唆し、こうした非対称的な状況の分析は現実の分析に不可欠であると考えられる。次に、現在採用している近視眼的な行動変更にforward-lookingな性質を加えることも大切であろう。実際の意思決定の場面では「近隣の都市が環境整備を行うかもしれない」という予想は自らの行動に影響をあたえる可能性があり、戦略的状況になっているかもしれない。この時、相手の出方に関する予想を持ち、そうした行動の帰結を予見する構造を想定する必要がある。今のところ具体的にどのようにこの発想をモデル化するかアイデアがないが、\citet{oyama09:aggl}など新経済地理学にforward-lookingな行動改訂を持ち込んだ研究を参考に試行してみる価値はあると思われる。

実証研究も行いたい。都市環境整備に関する経済学的な実証研究の蓄積が少ないことは第1節で指摘した。周辺領域の関連する研究およびその手法をも参考にしながら、今回のモデルで得られた知見を仮説として実証研究を行うことは有益であろう。一般に新経済地理学のモデルはデータにより検証することが難しいが、仮定を検証することなら可能なこともある。特に、今回導入した環境コストと環境整備の効果の定式化は観測される事実には矛盾しないものの、直接的にデータによる裏付けが先行研究にあるわけではないので検討したい。現在腹案として持っているのは、関東大震災が東京の中枢部を焼け野原にし、その後の復興でそこの都市環境整備が急速に進んだ事実を利用し、関東大震災前後の東京の土地価格を調べ統計分析を加えるというものである。最後に、今回のモデルが都市の中心部の環境整備の分析に適している保証はない。\footnote{分析を進める過程でモデルについて考えを巡らすたびに、このモデルは現実の分析に適した道具なのかと何度も自問自答したし、正直に言えば今も確信は持てないままでいる。}より適したモデルを、都市空間経済学の蓄積の中から探し出したり、あるいは自ら創造したりしていくという方針もありえるだろう。

\bibliography{my_ref}{}
\bibliographystyle{aea}

\end{document}